\section{Lesson 3}\label{sec:les_03}

In order to be able to compute the numerical solution of the one-dimension Neumann problem \eqref{eq:alg_form_Neumann}, we need to understand how $ V_h $ is defined (through its \textbf{basis functions} $ \varphi_j (x) $) and how it influences the \textbf{computation of the matrix} $ [A] $.

Beware that, from now on, the space discretizations will involve $ N_h+1 $ points (from $0$ to $N_h$), instead of $ N_h $ as in section \ref{sec:les_02}.

\subsection{Space discretization in 1D}

Let us consider a one-dimension space $ x \in (0,L) $ and the subspace $ V_h $ \eqref{eq:fe_subspace}.
We create a \textbf{mesh grid} using $ \dim{V_h} = N_h + 1 $ points, equally spaced within $ (0,L) $ with distance $ h $.
\begin{figure}[H]
    \centering
    \includegraphics[width=0.5\linewidth]{definition_Vh.png}
    \caption{Space discretization of domain $ (0,L) $ using $ N_h + 1 $ points}
\end{figure}

The chosen mesh has $ N_h + 1 $ points ($ x_j $) and $ N_h $ intervals ($ k_j $), formally defined as:
\begin{equation}\label{eq:mesh_1D}
    \begin{cases}
        x_j = x_0 + j h, \quad j = 0, \cdots, N_h \\[3pt]
        k_j = \left[ x_j, x_{j+1} \right], \quad j = 0, \cdots, N_h-1
    \end{cases}
    \quad (0,L) = \bigcup_j k_j
\end{equation}

We define the subspace $ V_h $ as the set of \textbf{piecewise linear polynomials}:
\begin{equation}\label{eq:piece_wise}
    V_h = \left\{ v : (0,L) \rightarrow \mathbb{R}, v \in V, v|_{k_j} \in \mathbb{P}^1 (k_j) \right\}
\end{equation}
\begin{figure}[H]
    \centering
    \includegraphics[width=0.5\linewidth]{piece_wise.png}
    \caption{Example of piecewise linear polynomial (blue)}
\end{figure}

\subsection{Basis functions of $ V_h $}

Each function $ v_h \in V_h $ is (by definition) a linear combination of the basis functions $ \varphi_j (x) $ of the subspace $ V_h $.
It is necessary to define them in the grid points $ x_j $ and we need $ N_h+1 $ functions (one for each point):
\begin{equation}
    \left\{ \varphi_j \right\}_{j=0}^{N_h} \text{ basis functions of } V_h, \quad \varphi_j (x_i) =
    \begin{cases}
        1, & i=j \\[3pt]
        0, & i \neq j
    \end{cases}
\end{equation}
\begin{figure}[H]
    \centering
    \includegraphics[width=0.5\linewidth]{basis_func.png}
    \caption{Basis functions $ \varphi_j $ at each grid point $ x_j $}
\end{figure}

Each function $ v_h \in V_h $ is a linear combination of the basis functions $ \varphi_j $:
\begin{equation}\label{eq:lin_comb}
    v_h (x) = \sum_{j=0}^{N_h} v_j \ \varphi_j (x), \quad v_h (x_i) = \sum_{j=0}^{N_h} v_j \ \varphi_j (x_i) = v_i 
\end{equation}

We can easily switch from finite element formulation to algebraic formulation using the \textbf{weights} $ v_j $ of $ v_h (x) $:
\begin{equation}
    v_h (x) \leftrightarrow \underline{v_j} = \left( v_0, \cdots, v_{N_h} \right) \in \mathbb{R}^{N_h+1}
\end{equation}

The basis functions are triangular in shape and defined as:
\begin{equation}
    \varphi_i (x) =
    \begin{cases}
        \dfrac{x-x_{i-1}}{h}, & x \in k_{i-1} \\[3pt]
        \dfrac{x_{i+1}-x}{h}, & x \in k_i \\[3pt]
        0, & \text{ otherwise}
    \end{cases}
\end{equation}

\begin{figure}[H]
    \centering
    \includegraphics[width=0.25\linewidth]{base.png}
    \caption{Basis function $ \varphi_i (x) \in V_h $}
\end{figure}

\subsection{Computation of $ [A] $}

We remind the definition of the matrix $ [A] $ from \eqref{eq:alg_form_Neumann} as (the dimension is $ N_h+1 $ due to the mesh definition):
\begin{equation}
    [A]_{ij} = a(\varphi_i, \varphi_j) = \int_0^L \rho(x) \ \varphi_i (x) \varphi_j (x) \ dx + \int_0^L \mu(x) \ \varphi_i' (x) \varphi_j' (x) \ dx, \quad \forall i,j = 0, \cdots, N_h
\end{equation}

It is generally possible to fix the ith row and span over the jth columns, though it is faster and more efficient to study the matrix entries diagonally.
The reason is that for each grid position $ x_i $ we only have three basis functions that interact between each other:
\begin{equation}
    [A]_{ij} = \int_{x_{i-1}}^{x_{i+1}} \rho(x) \ \varphi_i (x) \varphi_j (x) \ dx + \int_{x_{i-1}}^{x_{i+1}} \mu(x) \ \varphi_i' (x) \varphi_j' (x) \ dx \neq 0 \Leftrightarrow j = i-1, i, i+1
\end{equation}
\begin{figure}[H]
    \centering
    \includegraphics[width=0.5\linewidth]{comp_A.png}
    \caption{Basis functions of interest during the computation of $ [A]_{ij} $ ($ j = i-1, i, i+1 $) }
\end{figure}

We end up with three scenarios.
The first one is:
\begin{equation}
    [A]_{i, i-1} = \int_{x_{i-1}}^{x_{i}} \rho(x) \ \varphi_{i-1} (x) \varphi_i (x) \ dx + \int_{x_{i-1}}^{x_{i}} \mu(x) \ \varphi_{i-1}' (x) \varphi_i' (x) \ dx = 
\end{equation}
\begin{equation*}
    = \int_{x_{i-1}}^{x_{i}} \rho(x) \frac{x_i - x}{h} \frac{x - x_{i-1}}{h} \ dx + \int_{x_{i-1}}^{x_{i}} \mu(x) \left( - \frac{1}{h} \right) \frac{1}{h} \ dx
\end{equation*}
\begin{figure}[H]
    \centering
    \includegraphics[width=0.2\linewidth]{comp_A1.png}
    \caption{Computation of $ [A]_{i,i-1} $}    
\end{figure}

The second one is:
\begin{equation}
    [A]_{i, i} = \int_{x_{i-1}}^{x_{i}} \rho(x) \ \varphi_{i} (x) \varphi_i (x) \ dx + \int_{x_{i-1}}^{x_{i}} \mu(x) \ \varphi_{i}' (x) \varphi_i' (x) \ dx + 
\end{equation}
\begin{equation*}
    + \int_{x_{i}}^{x_{i+1}} \rho(x) \ \varphi_{i} (x) \varphi_i (x) \ dx + \int_{x_{i}}^{x_{i+1}} \mu(x) \ \varphi_{i}' (x) \varphi_i' (x) \ dx =
\end{equation*}
\begin{equation*}
    = \int_{x_{i-1}}^{x_i} \rho(x) \left(\frac{x - x_{i-1}}{h}\right)^2 dx + \int_{x_i}^{x_{i+1}} \rho(x) \left(\frac{x_{i+1} - x}{h}\right)^2 dx +
\end{equation*}
\begin{equation*}
    + \int_{x_{i-1}}^{x_{i}} \mu(x) \left( - \frac{1}{h} \right) \left( - \frac{1}{h} \right) \ dx + \int_{x_i}^{x_{i+1}} \mu(x) \frac{1}{h} \frac{1}{h} \ dx
\end{equation*}
\begin{figure}[H]
    \centering
    \includegraphics[width=0.25\linewidth]{comp_A2.png}
    \caption{Computation of $ [A]_{i,i} $}
\end{figure}

The third one is:
\begin{equation}
    [A]_{i, i+1} = \int_{x_{i}}^{x_{i+1}} \rho(x) \ \varphi_i (x) \varphi_{i+1} (x) \ dx + \int_{x_{i}}^{x_{i+1}} \mu(x) \ \varphi_i' (x) \varphi_{i+1}' (x) \ dx =
\end{equation}
\begin{equation*}
    = \int_{x_i}^{x_{i+1}} \rho(x) \frac{x_{i+1} - x}{h} \frac{x - x_i}{h} \ dx + \int_{x_{i-1}}^{x_{i}} \mu(x) \frac{1}{h} \left( - \frac{1}{h} \right) \ dx
\end{equation*}
\begin{figure}[H]
    \centering
    \includegraphics[width=0.2\linewidth]{comp_A3.png}
    \caption{Computation of $ [A]_{i, i+1} $}
\end{figure}

We may notice that, for each element on the diagonal $ [A]_{i,i} $, we have two other components along the same row (figure \ref{fig:assembly}).

\subsection{Reference element}

All entries are integrals over specific intervals $ k_{i-1}, k_i $ \eqref{eq:mesh_1D}, in which both an ascending and a descending segment may be present.
In order to obtain expressions as generalized as possible, we want to define a \textbf{reference element} of domain $ \hat{k} = (0,1) $ to map at each interval $ k_i $:
\begin{figure}[H]
    \centering
    \includegraphics[width=0.6\linewidth]{reference.png}
    \caption{Reference element and original interval}
    \label{fig:reference}
\end{figure}

The reference interval $ \hat{k} $ is function of $ \xi \in (0,1) $ (instead of $ x $) and the mapping $ \phi_i $ allows to get a specific interval:
\begin{equation}
    k_i = \phi_i (\hat{k}), \quad x = \phi_i (\xi) = x_i + \xi h \in (0,L)
\end{equation}
\begin{figure}[H]
    \centering
    \includegraphics[width=0.6\linewidth]{reference_mapping.png}
    \caption{Reference mapping}
\end{figure}

As pictured in figure \ref{fig:reference}, we want to express the physical basis functions $ \varphi_i $ in function of the reference basis functions $ \hat{\varphi}_0 $ and $ \hat{\varphi}_1 $:
\begin{equation}
    \hat{\varphi}_0 (\xi) = 1-\xi \overset{\phi_{i-1}}{\Longrightarrow} \varphi_{i-1} (x) = \hat{\varphi}_0 \left( \phi_{i-1}^{-1} (x) \right) = \hat{\varphi}_0 (\xi) = \hat{\varphi}_0 \left( \frac{x - x_{i-1}}{h} \right)
\end{equation}
\begin{equation}
    \hat{\varphi}_1 (\xi) = \xi \overset{\phi_{i-1}}{\Longrightarrow} \varphi_i (x) = \hat{\varphi}_1 \left( \phi_{i-1}^{-1} (x) \right) = \hat{\varphi}_1 (\xi) = \hat{\varphi}_1 \left( \frac{x - x_{i-1}}{h} \right)
\end{equation}

Finally, we can express all matrix entries $ [A]_{i,j} $ using the reference element ($ h $ in front of $ d\xi $ is the Jacobian of the transformation):
\begin{equation}
    [A]_{i, i-1} = \int_{x_{i-1}}^{x_{i}} \rho(x) \ \varphi_{i-1} (x) \varphi_i (x) \ dx + \int_{x_{i-1}}^{x_{i}} \mu(x) \ \varphi_{i-1}' (x) \varphi_i' (x) \ dx = 
\end{equation}
\begin{equation*}
    = \int_0^1 \rho(x) \ \hat{\varphi}_0 (\xi) \hat{\varphi}_1 (\xi) \ h d\xi + \int_0^1 \mu(x) \ \hat{\varphi}'_0 (\xi) \hat{\varphi}'_1 (\xi) \ h d\xi
\end{equation*}

\begin{equation}
    [A]_{i, i} = \int_{x_{i-1}}^{x_{i}} \rho(x) \ \varphi_{i} (x) \varphi_i (x) \ dx + \int_{x_{i-1}}^{x_{i}} \mu(x) \ \varphi_{i}' (x) \varphi_i' (x) \ dx + 
\end{equation}
\begin{equation*}
    + \int_{x_{i}}^{x_{i+1}} \rho(x) \ \varphi_{i} (x) \varphi_i (x) \ dx + \int_{x_{i}}^{x_{i+1}} \mu(x) \ \varphi_{i}' (x) \varphi_i' (x) \ dx =
\end{equation*}
\begin{equation*}
    = \int_0^1 \rho(x) \ \hat{\varphi}_0 (\xi) \hat{\varphi}_0 (\xi) \ h d\xi + \int_0^1 \mu(x) \ \hat{\varphi}'_0 (\xi) \hat{\varphi}'_0 (\xi) \ h d\xi +
\end{equation*}
\begin{equation*}
    + \int_0^1 \rho(x) \ \hat{\varphi}_1 (\xi) \hat{\varphi}_1 (\xi) \ h d\xi + \int_0^1 \mu(x) \ \hat{\varphi}'_1 (\xi) \hat{\varphi}'_1 (\xi) \ h d\xi
\end{equation*}

\begin{equation}
    [A]_{i, i+1} = \int_{x_{i}}^{x_{i-1}} \rho(x) \ \varphi_i (x) \varphi_{i+1} (x) \ dx + \int_{x_{i}}^{x_{i+1}} \mu(x) \ \varphi_i' (x) \varphi_{i+1}' (x) \ dx = 
\end{equation}
\begin{equation*}
    = \int_0^1 \rho(x) \ \hat{\varphi}_1 (\xi) \hat{\varphi}_0 (\xi) \ h d\xi + \int_0^1 \mu(x) \ \hat{\varphi}'_1 (\xi) \hat{\varphi}'_0 (\xi) \ h d\xi
\end{equation*}

The result is that we need to implement eight integrals to compute all entries of $ [A] $.
\begin{figure}[H]
    \centering
    \includegraphics[width=0.75\linewidth]{integrals.png}
    \caption{All integrals for three entries of $ [A] $}
\end{figure}

We are considering two intervals $ k_{i-1} $ and $ k_i $ (three grid points $  x_{i-1} $, $ x_i $ and $ x_{i+1} $), so each interval needs only four integrals.
\begin{figure}[H]
    \centering
    \includegraphics[width=0.9\linewidth]{integrals_2.png}
    \caption{All integrals for one interval $ k_i $}
\end{figure}

We may use the last consideration to loop over the diagonal elements to compute all entries, defining the \textbf{assembly} of $ [A] $.
\begin{figure}[H]
    \centering
    \includegraphics[width=0.4\linewidth]{assembly.png}
    \caption{Assembly of matrix [A], loop over element}
    \label{fig:assembly}
\end{figure}

\subsection{Quadrature rules}

The last step to compute the numerical solution of a one-dimension physical problem such as \eqref{eq:strong_form_Neumann} is to define a way to evaluate a generic integral:
\begin{equation*}
    \int_0^1 f(x) \ dx
\end{equation*}

We may use different \textbf{quadrature rules}, which are generally defined as (number of grid points $ N_q $ with weight $ w_q \in \mathbb{R} $ at position $ x_q \in (0,1) $):
\begin{equation}
    \int_0^1 f(x) \ dx \approx \sum_{j=1}^{N_q} w_q f(x_q)
\end{equation}

Using $ p+1 $ points, the maximum degree of precision is $ 2p+1 $.
\begin{figure}[H]
    \centering
    \includegraphics[width=0.35\linewidth]{quad_1.png}
    \includegraphics[width=0.35\linewidth]{quad_2.png}
    \includegraphics[width=0.35\linewidth]{quad_3.png}
    \includegraphics[width=0.35\linewidth]{quad_4.png}
    \caption{Quadrature rules (trapezoidal, mid-point, ?, Cavalieri-Simpson)}
    \label{fig:quad_rules}
\end{figure}

Trapezoidal rule:
\begin{equation}
    \int_0^1 f(x) \ dx = \frac{f(0) + f(1)}{2} (1-0) = \frac{1}{2} f(0) + \frac{1}{2} f(1)
\end{equation}

Mid-point rule:
\begin{equation}
    \int_0^1 f(x) \ dx = f \left( \frac{1}{2} \right) (1-0) = 1 \frac{1}{2} f(1)
\end{equation}

\clearpage
