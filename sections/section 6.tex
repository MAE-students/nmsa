\section{Lesson 6}\label{sec:les_06}

Up to section \ref{sec:les_05}, we studied simple wave equations (in both space and time) without any damping effect.
We want now to present the \textbf{PML conditions}, introduce \textbf{damped wave equations} and focus on \textbf{absorbing layers}.
We define the \textbf{Helmholtz equation} in the space-frequency domain to study this scenario.

\subsection{PML conditions}

Let's consider a one-dimensional space domain, divided in two: in the first part $ (-L,0) $ we have a simple wave equation, in the second part $ (0, \delta) $ we have a \textbf{damped wave equation}.
The wave equations that describe both cases are:
\begin{equation}
    \partial_{tt} u - \partial_{xx} u = 0, \quad \partial_{tt} u + \nu \ \partial_x u - \partial_x \left( \partial_x u + \nu^* \ \partial_{xt} u \right) = 0
\end{equation}

The term $ \nu \ \partial_x u $ describes \textbf{fluid friction}, while the term $ \nu^* \ \partial_{xt} u $ describes \textbf{viscous friction}.
The case we described is called \textbf{perfectly matched layer (PML)}.
\begin{figure}[H]
    \centering
    \includegraphics[width=0.65\linewidth]{pml.png}
    \includegraphics[width=0.2\linewidth]{pml_reflection.png}
    \caption{Acoustic problem with PML conditions (left), reflected wave $ R(\omega) $ and transmitted wave $ T(\omega) $ at the layer (right)}
\end{figure}

The general solution is:
\begin{equation}
    k^2 (\omega) = \frac{\omega^2 - j \omega \nu}{1 + j \omega \nu^*} \Rightarrow
    \begin{cases}
        \nu = \nu^* = 0, & \text{ for } x<0 \\[3pt]
        \nu, \nu^* > 0, & \text{ for } x>0
    \end{cases}
    \Rightarrow u(x,t) = A \ e^{j \left( \omega t - kx \right)} + B \ e^{j \left( \omega t + kx \right)}
\end{equation}

The last expression describes the sum of a \textbf{progressive wave} ($A$) and \textbf{regressive wave} ($B$).

\subsection{Absorbing layer}

A scenario we're interested in is avoiding reflections and maximizing transmissions into the layer (\textbf{absorbing layer}):
\begin{equation}
    u(x,t) = 
    \begin{cases}
        1 \ e^{j \left( \omega t - kx \right)} + R(\omega) \ e^{j \left( \omega t + kx \right)}, & x<0 \\[3pt]
        T(\omega) \ e^{j \left( \omega t - kx \right)}, & x>0
    \end{cases}
\end{equation}

In the left domain, we have a progressive incident wave with weight 1 and a regressive reflected wave with weight $ R(\omega) $.
In the right domain, we have a progressive transmitted wave with weight $ T(\omega) $.
The condition of continuity at the layer boundary $ x=0 $ imposes:
\begin{equation}
    \begin{cases}
        u(0^-, t) = u(0^+,t) \\[3pt]
        \partial_x u(0^-, t) = \partial_x u(0^+,t) + \nu^* \partial_{xt} u(0^+,t)
    \end{cases}
\end{equation}

As a result, we get reflection and transmission coefficients:
\begin{equation}
    R(\omega) = \frac{\omega - k \left( 1 + j \omega \nu^* \right)}{\omega + k \left( 1 + j \omega \nu^* \right)}, \quad \lim_{\nu^* \rightarrow + \infty} \abs{R(\omega)} = 1, \quad T(\omega) = 1 - R(\omega)
\end{equation}

The previous limit suggests that the more damping we have (high $ \nu^* $), the more reflection we may observe.
This result is not desirable, so let's try a different approach: mass density $ \rho (x) $ and horizontal tension $ \mu (x) $ are not constants anymore (equal to 1 previously in the current section \ref{sec:les_06}), and have generally different values in the two space domains.
\begin{equation}
    \rho (x) \ \partial_{tt} u - \partial_x \left( \mu (x) \ \partial_x u \right) = 0, \quad c(x) = \sqrt{\frac{\mu (x)}{\rho (x)}}, \quad Z(x) = \sqrt{\rho (x) \mu (x)}
\end{equation}

We introduce the concept of \textbf{impedance} $ Z(x) $ that describes the specific space domain (or position) characteristics.
Both space domains $ (-L,0) $ and $ (0,\delta) $ have specific constant values of wave speed $ c $ and impedance $ Z $:
\begin{figure}[H]
    \centering
    \includegraphics[width=0.5\linewidth]{pml_impedances.png}
    \caption{PML conditions with impedances ($Z$, $Z^*$) and wave speeds ($c$, $c^*$)}
\end{figure}

As we had before:
\begin{equation}
    u(x,t) = 
    \begin{cases}
        1 \ e^{j \left( \omega t - kx \right)} + R(\omega) \ e^{j \left( \omega t + kx \right)}, & x<0 \\[3pt]
        T(\omega) \ e^{j \left( \omega t - kx \right)}, & x>0
    \end{cases}
\end{equation}

The condition of continuity at the layer boundary $ x=0 $ imposes:
\begin{equation}
    \begin{cases}
        u(0^-, t) = u(0^+,t) \\[3pt]
        \mu^- \ \partial_x u(0^-, t) = \mu^+ \ \partial_x u(0^+,t)
    \end{cases}
\end{equation}

Where $ \mu^- $ and $ \mu^+ $ are associated to left and right space domain, respectively.
As a result, we get reflection and transmission coefficients:
\begin{equation}
    R(\omega) = \frac{Z - Z^*}{Z + Z^*}, \quad T(\omega) = \frac{2Z}{Z + Z^*}
\end{equation}

The goal is always the same: avoiding reflections and maximizing transmissions.
\begin{equation}\label{eq:imp_matching}
    R(\omega) = 0, \ T(\omega) = 1 \Leftrightarrow Z = Z^*
\end{equation}

The last expression is called \textbf{impedance matching}.

\subsection{Helmholtz equation}

The question now is: how can we implement an absorbing layer that works as we desire?

Let's consider the following form for both solution $ u(x,t) $ and force $ f(x,t) $:
\begin{equation}
    u(x,t) = \hat{u}(x) e^{j \omega t}, \quad f(x,t) = \hat{f}(x) e^{j \omega t}
\end{equation}

Applied into the wave equation \eqref{eq:wave_1D}, it becomes:
\begin{equation} \label{eq:helmholtz}
    \rho (x) \ \partial_{tt} u - \partial_x \left( \mu (x) \ \partial_x u \right) = f \Rightarrow \rho (x) \left( j \omega \right)^2 \hat{u}(x) e^{j \omega t} - \partial_x \left( \mu (x) \ \partial_x \hat{u}(x) \ e^{j \omega t} \right) = \hat{f}(x) \ e^{j \omega t}
\end{equation}
\begin{equation*}
    \Rightarrow - \rho (x) \omega^2 \hat{u}(x) \ \cancel{e^{j \omega t}} - \partial_x \left( \mu (x) \ \partial_x \hat{u}(x) \ \cancel{e^{j \omega t}} \right) = \hat{f}(x) \ \cancel{e^{j \omega t}}
    \Rightarrow - \rho (x) \omega^2 \hat{u}(x) - \partial_x \left( \mu (x) \ \partial_x \hat{u}(x) \right) = \hat{f}(x)
\end{equation*}

The last expression is called \textbf{Helmholtz equation} and describes the problem in the space-frequency domain.
The function $ u(x,t) $ is solution of the wave equation \eqref{eq:wave_1D} \textbf{if and only if} the function $ \hat{u} (x,\omega) $ is solution of \eqref{eq:helmholtz}.

We can switch between time and frequency domain using the Fourier transform:
\begin{equation}
    u(x,t) = \frac{1}{\sqrt{2\pi}} \int_{\mathbb{R}} e^{j \omega t} \ \hat{u}(x,\omega) \ d\omega \approx \sum_{n=1}^{N^*} C_n \ e^{j \omega_n t} \ \hat{u}(x, \omega_n)
\end{equation}
\begin{equation*}
    \hat{u}(x,\omega) = \frac{1}{\sqrt{2\pi}} \int_{\mathbb{R}} e^{-j \omega t} \ u(x,t) \ dt
\end{equation*}

We want to express the impedance matching condition \eqref{eq:imp_matching} in frequency domain:
\begin{equation}
    \begin{cases}
        \rho (x,\omega) = \rho (x), \quad \mu (x,\omega) = \mu (x), & x<0 \\[3pt]
        \rho (x,\omega) = \rho^* (x), \quad \mu (x,\omega) = \mu^* (x), & x \geq 0
    \end{cases}
\end{equation}
\begin{figure}[H]
    \centering
    \includegraphics[width=0.5\linewidth]{pml_imp_match.png}
    \caption{Impedance matching for absorbing layers in frequency domain}
\end{figure}

The impedance matching condition \eqref{eq:imp_matching} states that:
\begin{equation}
    Z(0^-, \omega) = Z(0^+, \omega) \Rightarrow \sqrt{\rho(0^-) \mu(0^-)} = \sqrt{\rho^*(0^+) \mu^*(0^+)} 
\end{equation}

Let's introduce the coefficient $ d(\omega) \in \mathbb{C} $ to describe impedance matching and amplitude reduction for $ x>0 $:
\begin{equation}
    \rho^*(x) = \frac{\rho(x)}{d(\omega)}, \quad \mu^*(x) = \mu(x) \ d(\omega) 
\end{equation}

Applied into the Helmholtz equation \eqref{eq:helmholtz}, it becomes:
\begin{equation}
    - \frac{\rho(x)}{d(\omega)} \omega^2 \hat{u}(x) - \partial_x \left( \mu(x) \ d(\omega) \ \partial_x \hat{u}(x) \right) = 0
\end{equation}
\begin{equation*}
    d(\omega) = \frac{1}{a + jb} \Rightarrow u(x,t) = e^{j \omega \left( t \pm \frac{ax}{c} \right)} \ e^{\mp \omega \frac{bx}{c}}
\end{equation*}

The first term describes the propagating wave, the second term describes the damping effect.

Let's consider $ a=1 $ and $ b = - \dfrac{\sigma}{\omega} $ ($ \sigma > 0 $, considering that $ \omega b < 0 $):
\begin{equation}
    u(x,t) = e^{j \omega \left( t \pm \frac{x}{c} \right)} \ e^{\pm \frac{\sigma x}{c}}, \quad d(\omega) = \frac{1}{1 - j \frac{\sigma}{\omega}} = \frac{j\omega}{\sigma + j\omega} \text{ constant}
\end{equation}

The first term of $ u(x,t) $ suggests that the phase of the wave is constant, while the second term suggests that the minimum length $ L $ of the layer necessary to entirely attenuate the waves is:
\begin{equation}
    e^{\pm \frac{\sigma x}{c}} |_{x=L} = e^{\pm \frac{\sigma}{c} L} \Rightarrow L = \frac{c}{\sigma}
\end{equation}
\begin{figure}[H]
    \centering
    \includegraphics[width=0.45\linewidth]{pml_layer_length.png}
    \caption{Minimum absorbing layer length}
\end{figure}

The resulting Helmholtz equation (in the \textbf{space-frequency} domain):
\begin{equation}
    \rho (x) \left( \sigma + j \omega \right) \hat{u} - \partial_x \left( \mu (x) \frac{1}{\sigma + j \omega} \partial_x \hat{u} \right) = 0, \quad x \geq 0
\end{equation}

Becomes, in the \textbf{space-time} domain:
\begin{equation}
    \rho \ \partial_{tt} u + 2 \sigma \ \partial_t u + \sigma^2 u - \partial_x \left( \mu \ \partial_x u \right) = 0, \quad x \geq 0
\end{equation}

Let's go back to \eqref{eq:helmholtz}, with $ \rho (x) $ and $ \mu (x) $ as constants and adding boundary conditions:
\begin{equation}
    \begin{cases}
        \rho \omega^2 \ \hat{u}(x) + \mu \ \partial_{xx} \hat{u}(x) = \hat{f}(x) \\[3pt]
        \hat{u} (0) = \hat{u} (L) = 0
    \end{cases}
    \Rightarrow
    \begin{cases}
        \cancel{\rho \omega^2} \ \hat{u}(x) + \frac{\mu}{\rho \omega^2} \partial_{xx} \hat{u}(x) = \frac{\hat{f}(x)}{\rho \omega^2} \\[3pt]
        \hat{u} (0) = \hat{u} (L) = 0
    \end{cases}
\end{equation}
\begin{equation*}
    \Rightarrow
    \begin{cases}
        \hat{u}(x) + \frac{1}{k^2} \partial_{xx} \hat{u}(x) = \hat{g}(x) \\[3pt]
        \hat{u} (0) = \hat{u} (L) = 0
    \end{cases}
\end{equation*}

The solution of $ \hat{g}(x) = 0 $ as the following form:
\begin{equation}
    \hat{u}(x) = C_1 e^{j kx} + C_2 e^{-j kx}
\end{equation}

Applying the boundary conditions:
\begin{equation}
    \hat{u} (0) = 0 \Rightarrow C_1 + C_2 = 0 \Rightarrow C_2 = -C_1
\end{equation}
\begin{equation*}
    \hat{u} (L) = 0 \Rightarrow C_1 \left( e^{jkL} - e^{-jkL} \right) = C_1 \ 2j \sin \left( kL \right) = 0 \
\end{equation*}
\begin{equation*}
    \Rightarrow \sin \left( kL \right) = 0 \Rightarrow kL = n \pi \Rightarrow k_n = n \frac{\pi}{L}, \quad n = 1, 2, \cdots
\end{equation*}

The resulting solution is:
\begin{equation}
    \hat{u}(x) = \sum_{n=1}^{+\infty} \hat{u}_n \sin \left( k_n x \right) = \sum_{n=1}^{+\infty} \hat{u}_n \sin \left( n \frac{\pi}{L} x \right)
\end{equation}

Expanding the function $ \hat{g}(x) $ by Fourier series:
\begin{equation}
    \hat{g}(x) = \sum_{n=1}^{+\infty} \hat{g}_n \sin \left( n \frac{\pi}{L} x \right)
\end{equation}

The previous Helmholtz equation becomes:
\begin{equation}
    \hat{u}(x) + \frac{1}{k^2} \partial_{xx} \hat{u}(x) = \hat{g}(x) \Rightarrow \sum_{n=1}^{+\infty} \hat{u}_n \left[ 1 - \frac{1}{k_n^2} \left( n \frac{\pi}{L} \right)^2 \right] \sin \left( n \frac{\pi}{L} x \right) = \sum_{n=1}^{+\infty} \hat{g}_n \sin \left( n \frac{\pi}{L} x \right)
\end{equation}
\begin{equation*}
    \Rightarrow \hat{u}_n = \frac{\hat{g}_n}{1 - \left( \frac{n \pi}{k_n L} \right)^2}
\end{equation*}

The last expression is the generic coefficient of the Fourier series of $ \hat{u}(x) $.

Let's make some observations:
\begin{itemize}
    \item $ \frac{n \pi}{k_n L} \neq \pm 1, \ \forall n \Rightarrow \hat{u}_n \text{ well defined} $
    \item $  \exists \ n^* \text{ so that } \left( \frac{n \pi}{k_n L} \right)^2 = 1 \Rightarrow \hat{u}_{n^*} \text{ not defined} \Rightarrow
    \begin{cases}
        \hat{g}_{n^*} = 0 \Rightarrow \infty \text{ solutions, } k_{n^*} = \frac{\omega_{n^*}}{c} \text{ leads to a resonance} \\[3pt]
        \hat{g}_{n^*} \neq 0 \Rightarrow \text{no solutions}
    \end{cases}
    $
\end{itemize}

For $ c=1 $ we have $ \omega_n^2 = k_n^2 $, where $ k_n $ are the eigenvalues of the following:
\begin{equation}
    \partial_{xx} \hat{u}(x) + k_n^2 \ \hat{u}(x) = 0, \quad k_n^2 = \left( \frac{n \pi}{L} \right)^2
\end{equation}
\begin{figure}[H]
    \centering
    \includegraphics[width=0.45\linewidth]{resonances.png}
    \caption{Resonances in the frequency domain}
\end{figure}

The solution is infinite at resonances, without any damping effect, so we need to consider a damping factor $ \gamma $:
\begin{equation}
    \rho \ \partial_{tt} u - \partial_x \left( \mu \ \partial_x u \right) + \gamma \ \partial_t u = 0, \quad \gamma > 0
\end{equation}
\begin{equation*}
    \Rightarrow - \rho \omega^2 \ \hat{u}(x) - \partial_x \left( \mu \ \partial_x \hat{u} \right) -j \omega \gamma \ \hat{u}(x) = 0
\end{equation*}

Considering $ \rho $ and $ \mu $ constants:
\begin{equation}
    - \partial_{xx} \hat{u}(x) = \left( \frac{\rho \omega^2}{\mu} + j \frac{\omega \gamma}{\mu} \right) \hat{u}(x) = \left( k^2  + j \frac{\omega \gamma}{\mu} \right) \hat{u}(x)
\end{equation}

The eigenvalues of the damped problem are not purely real anymore (equal to $k_n$), but have an imaginary part that prevents divisions by zero (resonance).

We may apply different boundary conditions to this problem:
\begin{itemize}
    \item sound soft: $ \hat{u}(x) = g_D(x) $
    \item sound hard: $ \frac{j}{\rho \omega} \ \partial_x \hat{u} \ n(x) = g_N(x), \quad x=\left\{0,L\right\}, \quad n(x) =
    \begin{cases}
        -1, & x=0 \\[3pt]
        1, & x=L
    \end{cases}
    $
    \item impedance: $ \hat{u}(x) = Z(x) \left( \dfrac{j}{\rho \omega} \ \partial_x \hat{u} \ n(x) \right), \quad x=\left\{0,L\right\}, \quad Z(x) = 1 \text{ absorbing condition} $
\end{itemize}

In acoustic problems, we may identify $ \hat{u}(x) $ as the pressure $ p(x) $, and the quantity $ \dfrac{j}{\rho \omega} \ \partial_x \hat{u} $ as the particle velocity $ v(x) $.

\clearpage
