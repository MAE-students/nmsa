\section{Lesson 7}\label{sec:les_07}

In this lesson, we want to study \textbf{Helmholtz problems}, evaluate the error forms of \textbf{dissipation} and \textbf{dispersion} that we may have when computing a wave equation, and introduce an alternative to the finite element method called \textbf{spectral element method}.

\subsection{Helmholtz problem}

Let's consider the following problem and boundary conditions in $ x \in (0,L) $:
\begin{equation}\label{eq:Helmholtz_prob}
    \begin{cases}
        \partial_{xx} u(x) + k^2 u(x) = f(x) \\[3pt]
        \partial_x u(0) = g_1 \\[3pt]
        \partial_x u(L) = 0
    \end{cases}
\end{equation}

The Helmholtz problem \eqref{eq:Helmholtz_prob} is characterized by boundary velocity values.
\begin{figure}[H]
    \centering
    \includegraphics[width=0.45\linewidth]{Helmholtz_problem.png}
    \caption{Boundary velocities for the Helmholtz problem}
\end{figure}

The corresponding weak formulation starts, as usual, from:
\begin{equation}
    \int_0^L \partial_{xx} u(x) \ v(x) \ dx + \int_0^L k^2 u(x) \ v(x) \ dx = \int_0^L f(x) v(x) \ dx
\end{equation}

Solving the first integral by parts:
\begin{equation}
    \int_0^L \partial_{xx} u(x) \ v(x) \ dx = \left[ \partial_x u(x) \ v(x) \right]_0^L - \int_0^L \partial_x u(x) \  \partial_x v(x) \ dx =
\end{equation}
\begin{equation*}
    = \cancelto{0}{\partial_x u(L)} \ v(L) - \cancelto{g_1}{\partial_x u(0)} \ v(0) - \int_0^L \partial_x u(x) \  \partial_x v(x) \ dx = - g_1 v(0) - \int_0^L \partial_x u(x) \  \partial_x v(x) \ dx
\end{equation*}

Result, the \textbf{weak formulation} for \eqref{eq:Helmholtz_prob} states that $ u(x) \in H^1(0,L) $ is solution of ($ \forall v \in H^1(0,L) $):
\begin{equation}
    - \int_0^L \partial_x u(x) \  \partial_x v(x) \ dx + \int_0^L k^2 u(x) \ v(x) \ dx = \int_0^L f(x) v(x) \ dx + g_1 v(0)
\end{equation}

As usual, the problem is well-posed thanks to the Lax-Milgram lemma.
The only difference with respect to all previous cases is the $ - $ sign at the first integral.

Regarding the finite element method, we remind the definition of the subspace $ V_h $ at \eqref{eq:piece_wise}.
The \textbf{finite element formulation} (or Galerkin formulation) for \eqref{eq:Helmholtz_prob} states that $ u_h \in V_h $ is solution of ($ \forall v_h \in V_h $):
\begin{equation}
    - \int_0^L \partial_x u_h(x) \  \partial_x v_h(x) \ dx + \int_0^L k^2 u_h(x) \ v_h(x) \ dx = \int_0^L f(x) v_h(x) \ dx + g_1 v_h(0)
\end{equation}
\begin{equation*}
    \Leftrightarrow - a(u_h,v_h) + m(u_h,v_h) = F(v_h)
\end{equation*}

The \textbf{algebraic formulation} is obtained, as usual, following the logic of \eqref{eq:alg_form_Neumann}, which is defining $ u_h \in V_h $ as linear combination of basis functions and $ v_h = \varphi_i $ as in \eqref{eq:Vh_basis}:
\begin{equation}
    u_h(x) = \sum_{j=0}^{N_h} u_j \ \varphi_j(x), \quad v_h(x) = \varphi_i(x), \quad \forall i = 0, \cdots, N_h
\end{equation}
\begin{equation*}
    \sum_{j=0}^{N_h} u_j \ \left[ m(\varphi_j, \varphi_i) - a(\varphi_j, \varphi_i) \right] = F(\varphi_i) \Rightarrow \left( [M] - [A] \right) \underline{u} = \underline{F}
\end{equation*}

When $ \underline{F} = \underline{0} $, we have an eigenvalue-eigenvector problem:
\begin{equation}
    \left( [M] - [A] \right) \underline{u} = \underline{0} \Leftrightarrow [M] \ \underline{u} = [A] \ \underline{u} \Leftrightarrow k^2 \ [\hat{M}] \ \underline{u} = [A] \ \underline{u}
\end{equation}

By approximating the eigenvalue $k$, we approximate the frequency $\omega$ too:
\begin{equation}
    k^2 = \frac{\omega^2}{c^2}, \quad k_h \approx k \Rightarrow \omega_h \approx \omega
\end{equation}

We can compute a verification test:
\begin{equation}
    \begin{cases}
        \partial_{xx} u(x) + k^2 u(x) = f(x), & x \in (0,1) \\[3pt]
        \partial_x u(0) = - \rho \omega^2 X, & \rho = 1.225 \text{ kg/m}^3, \quad X = 10^{-3} \\[3pt]
        \partial_x u(1) = 0, & \omega = \{200 \pi, 500 \pi\} \text{ rad/s}
    \end{cases}
\end{equation}

To compute the modes $ \omega_i $ when $ \partial_x u(0) = 0 $:
\begin{equation}
    [A] \ \underline{u} = k^2 \ [\hat{M}] \ \underline{u}, \quad
    \begin{cases}
        \partial_{xx} u(x) + k^2 u(x) = 0 \\[3pt]
        \partial_x u(0) = \partial_x u(L) = 0
    \end{cases}
\end{equation}
\begin{equation*}
    \lambda_{1,2} = \pm jk \Rightarrow u(x) = C_1 e^{j kx} + C_2 e^{-j kx} \Rightarrow \omega_n = n \frac{\pi}{L}
\end{equation*}

\subsection{Dissipation and dispersion}

When we switch from continuous to discrete domain, we may observe some differences between \textbf{analytic} and \textbf{approximated} solutions.
The dissipation error consists in amplitude reduction, while dispersion consists in a stretch along the space.
\begin{figure}[H]
    \centering
    \includegraphics[width=0.6\linewidth]{Dissipation_dispersion.png}
    \caption{Dissipation and dispersion effects}
\end{figure}

Let's study the wave equation in an infinite one-dimensional space domain, with no external sources and constant material properties ($ \mu $ and $ \rho $):
\begin{equation}
    \begin{cases}
        \partial_{tt} u - c^2 \partial_{xx} u = 0, \quad x \in \mathbb{R}, \quad c = \sqrt{\frac{\mu}{\rho}} \\[3pt]
        u(x,0) = u_0(x) \\[3pt]
        \partial_t u(x,0) = v_0(x)
    \end{cases}
    \Rightarrow
    u(x,t) = e^{j \left( \omega t - kx \right)}, \quad k = \frac{2 \pi}{\lambda}, \quad \omega = \frac{2 \pi}{T}
\end{equation}

Substituting the solution $ u(x,t) $ into the wave equation, we obtain the \textbf{dispersion relation}:
\begin{equation}
    \partial_{tt} u = - \omega^2 u(x,t), \quad \partial_{xx} u = - k^2 u(x,t)
\end{equation}
\begin{equation*}
    \Rightarrow - \omega^2 u(x,t) + c^2 k^2 u(x,t) = \left( -\omega^2 + c^2 k^2 \right) u(x,t) = 0 \Rightarrow \omega^2 = c^2 k^2
\end{equation*}

The next step is to discretize the infinite space domain (mesh).
\begin{figure}[H]
    \centering
    \includegraphics[width=0.6\linewidth]{Inifinte_mesh.png}
    \caption{Mesh for an infinite space domain}
\end{figure}

As usual, the \textbf{algebraic formulation} is:
\begin{equation}\label{eq:alg_infinite}
    \begin{cases}
        [M] \ \ddot{\underline{u}} + [A] \ \underline{u} = \underline{0}, \quad t>0 \\[3pt]
        \text{+ initial conditions}
    \end{cases}
\end{equation}

The spatial mesh has infinite points, so we have infinite basis functions $ \varphi_i $: the problem of having \textbf{infinite vectors} (due to the infinite dimension of the problem) is clear.

The generic equation of \eqref{eq:alg_infinite} (i-th row, $ v_h = \varphi_i $) is finite in dimension and takes into account the presence of $ \varphi_i $, $ \varphi_{i-1} $ and $ \varphi_{i+1} $ in the space interval $ \left( -h, h \right) $:
\begin{equation}
    m_{i,i-1} \ddot{u}_{i-1}(t) + m_{i,i} \ddot{u}_i(t) + m_{i,i+1} \ddot{u}_{i+1}(t) = -c^2 \left[ a_{i,i-1} u_{i-1}(t) + a_{i,i} u_{i-1}(t) + a_{i,i+1} u_{i+1}(t) \right]
\end{equation}
\begin{equation*}
    \frac{h}{6} \ddot{u}_{i-1}(t) + \frac{2h}{3} \ddot{u}_i(t) + \frac{h}{6} \ddot{u}_{i+1}(t) = -c^2 \left[ -\frac{1}{h} u_{i-1}(t) + \frac{2}{h} u_{i-1}(t) - \frac{1}{h} u_{i+1}(t) \right]
\end{equation*}

We define the analytic solution in the mesh nodes only:
\begin{equation}
    u_i = u(x_i,t) = e^{j \left( \omega t - kx_i \right)}, \quad \ddot{u}_i = \ddot{u}(x_i,t)
\end{equation}
\begin{equation*}
    u_{i+1} = e^{j \left( \omega t - kx_{i+1} \right)} = e^{j \left( \omega t - kx_i - kh \right)} = u_i e^{-jkh}
\end{equation*}
\begin{equation*}
    u_{i-1} = e^{j \left( \omega t - kx_{i-1} \right)} = e^{j \left( \omega t - kx_i + kh \right)} = u_i e^{jkh}
\end{equation*}

The generic row of the system \eqref{eq:alg_infinite} is:
\begin{equation}
    \left( \frac{h}{6} e^{jkh} + \frac{2h}{3} + \frac{h}{6} e^{-jkh} \right) \ddot{u}_i = -c^2 \left( -\frac{1}{h} e^{jkh} + \frac{2}{h} - \frac{1}{h} e^{-jkh} \right) u_i
\end{equation}
\begin{equation*}
    h^2 \left( \frac{1}{6} e^{jkh} + \frac{2}{3} + \frac{1}{6} e^{-jkh} \right) \ddot{u}_i = -c^2 \left( - e^{jkh} + 2 - e^{-jkh} \right) u_i
\end{equation*}

We still have functions of time, so we apply time discretization (\textbf{central difference}):
\begin{equation}
    u_i^n = u(x_i,t_n) = e^{j \left( \omega t_n - kx_i \right)}
\end{equation}
\begin{equation*}
    u_i^{n+1} = e^{j \left( \omega t_{n+1} - kx_i \right)} = e^{j \left( \omega t_n + \omega \Delta t - kx_i \right)} = u_i^n e^{j \omega \Delta t}
\end{equation*}
\begin{equation*}
    u_i^{n-1} = e^{j \left( \omega t_{n-1} - kx_i \right)} = e^{j \left( \omega t_n - \omega \Delta t - kx_i \right)} = u_i^n e^{-j \omega \Delta t}
\end{equation*}

Using the 2nd order Taylor expansion:
\begin{equation}
    \ddot{u}_i^n = \ddot{u}_i (t_n) \approx \frac{u_i^{n+1} - 2 u_i^n + u_i^{n-1}}{\Delta t^2} = \frac{e^{j\omega \Delta t} - 2 + e^{-j \omega \Delta t}}{\Delta t^2} u_i^n
\end{equation}

The generic row of the system \eqref{eq:alg_infinite} becomes:
\begin{equation}
    h^2 \left( \frac{1}{6} e^{jkh} + \frac{2}{3} + \frac{1}{6} e^{-jkh} \right) \left( \frac{e^{j\omega \Delta t} - 2 + e^{-j \omega \Delta t}}{\Delta t^2} \right) u_i^n = -c^2 \left( - e^{jkh} + 2 - e^{-jkh} \right) u_i^n
\end{equation}

So, in order to have solution $ u_h(x,t) $, the following must be valid:
\begin{equation}
    h^2 \left( \frac{1}{6} e^{jkh} + \frac{2}{3} + \frac{1}{6} e^{-jkh} \right) \left( \frac{e^{j\omega \Delta t} - 2 + e^{-j \omega \Delta t}}{\Delta t^2} \right) \cancel{u_i^n} = -c^2 \left( - e^{jkh} + 2 - e^{-jkh} \right) \cancel{u_i^n}
\end{equation}
\begin{equation*}
    h^2 \left( \frac{e^{jkh} + e^{-jkh}}{6} + \frac{2}{3} \right) \left( \frac{e^{j \omega \Delta t} + e^{-j \omega \Delta t}}{\Delta t^2} - \frac{2}{\Delta t^2} \right) = c^2 \left( e^{jkh} + e^{-jkh} - 2 \right)
\end{equation*}
\begin{equation*}
    h^2 \left( \frac{\cos \left( kh \right) + 2}{3} \right) \left( \frac{2 \cos \left( \omega \Delta t \right) - 2}{\Delta t^2} \right) = c^2 \left( 2 \cos \left( kh \right) - 2 \right)
\end{equation*}
\begin{equation*}
    \Rightarrow \frac{4}{\Delta t^2} \sin \left( \frac{\omega \Delta t}{2} \right) = 3 \frac{c^2}{h^2} \frac{2 - 2 \cos \left( kh \right)}{2 + 2 \cos \left( kh \right)}
\end{equation*}

We obtained the \textbf{dispersion relation} at the discrete level.
The continuous level relation is obtained using limit operations:
\begin{equation}
    \lim_{\omega \Delta t \rightarrow 0} \sin \left( \frac{\omega \Delta t}{2} \right) = \omega^2, \quad \lim_{kh \rightarrow 0} \frac{2 - 2 \cos \left( kh \right)}{2 + 2 \cos \left( kh \right)} = k^2 \Rightarrow \omega^2 \approx c^2 k^2
\end{equation}

In order to have an accurate finite element approximation, we generally want:
\begin{itemize}
    \item $ \omega \Delta t \rightarrow 0 \Rightarrow \omega \nearrow, \ \Delta t << 1 $
    \item $ kh \rightarrow 0 \Rightarrow k \nearrow, \ h << 1 $
\end{itemize}

As a result, the finite element method approximation does not introduce dispersion errors whenever the discretization (mesh size $h$, time step $\Delta t$) gets more accurate with increasing $\omega$ and $k$.
The discretization accuracy gets better with decreasing $h$ and $\Delta t$, and goes at cost of computational resources (more points to evaluate both in space and time).

\subsection{Spectral Element Method}

In some cases, the finite element method is limitative to study certain problems, that is way we introduce methods with higher orders.
We define special basis functions, we apply them into not-equispaced grid nodes and specify the quadrature formulas for said grid nodes.

Let's define the interval $ I=[-1,1] $ and the \textbf{Legendre polynomials} in said interval (by recursion):
\begin{equation}\label{eq:legendre}
    \begin{cases}
        L_0 = 1, \quad L_1 = x \\[3pt]
        L_{k+1} = \dfrac{2k+1}{k+1} L_k \ x - \dfrac{k}{k+1} L_{k-1}
    \end{cases}
    , \quad k = 1, 2, \cdots
\end{equation}

The set of Legendre polynomials $ \{L_k\}_{k=0}^{\infty} $ is \textbf{orthogonal} in $I$ (with respect to the scalar product):
\begin{equation}
    \int_{-1}^{1} L_m(x) L_n(x) \ dx =
    \begin{cases}
        0, & m \neq n \\[3pt]
        \left( m + \frac{1}{2} \right)^{-1}, & m = n
    \end{cases}
\end{equation}

This basis functions need a \textbf{suitable grid} (not-equispaced nodes).
Specifically, we specify the grid nodes (in $I$) where $ L_n'(x)=0 $.
In this way, we define local grid nodes called \textbf{Gauss-Legendre-Lobatto points} (or GLL points).
\begin{figure}[H]
    \centering
    \includegraphics[width=0.55\linewidth]{GLL_grid.png}
    \caption{GLL points}
\end{figure}

Starting from this grid, we define the \textbf{basis functions} $ \varphi_i \in \mathbb{P}^n $ ($ n+1 $ polynomials) as:
\begin{equation}
    \varphi_i(x) = -\frac{1}{n (n+1)} \frac{i-x^2}{x-x_i} \frac{L_n'(x)}{L_n(x_i)}, \quad i = 0, \cdots, n
\end{equation}
\begin{equation*}
    \varphi_i(x_j) = -\frac{1}{n (n+1)} \frac{i-x^2}{x_j-x_i} \frac{L_n'(x_j)}{L_n(x_i)} = \delta_{ij}, \quad i,j = 0, \cdots, n
\end{equation*}
\begin{figure}[H]
    \centering
    \includegraphics[width=0.75\linewidth]{SEM-examples.png}
    \caption{Examples of basis functions}
\end{figure}

The \textbf{quadrature rule}, with the GLL points, is (weights $ \alpha_i $, grid nodes $ x_i $):
\begin{equation}
    \int_{-1}^{1} f(x) \ dx \approx \sum_{i=1}^m \alpha_i \ f(x_i), \quad \alpha_i = \frac{2}{i (i+1)} \frac{1}{L_n^2(x_i)} > 0
\end{equation}

The GLL rule with n+1 points is exact for polynomials up to order 2n-1.

Let's now apply this spectral element method to the wave equation:
\begin{equation}
    \begin{cases}
        \rho \ \partial_{tt} u - \partial_x \left( \mu \ \partial_x u \right) = f(x,t) \\[3pt]
        \mu \ \partial_x u(0) = \mu \ \partial_x u(L) = 0 \\[3pt]
        u(x,0) = u_0(x) \\[3pt]
        \partial_t u(x,0) = v_0(x)
    \end{cases}
    \quad
    \begin{array}{c}
        x \in (0,L) \\[3pt]
        t \in (0,T]
    \end{array}
\end{equation}

The \textbf{Galerkin formulation} states that $ u_h \in V_h^n \subset H^1(0,L) $ is solution of:
\begin{equation}
    \int_0^L \rho \ \partial_{tt} u_h \ v_h \ dx + \int_0^L \mu \ \partial_x u_h \ \partial_x v_h \ dx = \int_0^L f v_h \ dx, \quad \forall v_h \in V_h^n
\end{equation}

Note that the subspace $ V_h $ is extended to include $ \mathbb{P}^n $ (not only $ \mathbb{P}^1 $ as in \eqref{eq:piece_wise}):
\begin{equation}
    V_h^n = \left\{ v \in C^0(0,L): v|_{k_j} \in \mathbb{P}^n (k_j), \forall j \right\}
\end{equation}

This new subspace is defined by the basis functions $ \psi_j $, where their number $N$ depends on the polynomial degree $n$:
\begin{equation}
    \dim (V_h^n) = N = (n+1) M - (M-1) = nM + M - M + 1 = nM + 1
\end{equation}
\begin{figure}[H]
    \centering
    \includegraphics[width=0.5\linewidth]{V_h_n_basis.png}
    \caption{Dimension $ \dim (V_h^n) $ computation with $ M=6 $ and $ n=3 $}
\end{figure}

The term $ (M-1) $ takes into account the fact that basis functions between intervals are glued together to keep continuity.

After the discretization, the \textbf{algebraic formulation} is the same and $ [M] $ is diagonal $ \forall V_h^n $:
\begin{equation}
    [M] \ \ddot{\underline{u}} + [A] \ \underline{u} = \underline{F}
\end{equation}
\begin{equation*}
    [M]_{ij} = \int_0^L \rho \ \psi_j (x) \psi_i (x) \ dx = \sum_{m=1}^{M} \int_{k_m} \rho \ \psi_j (x) \psi_i (x) \ dx = \sum_{m=1}^{M} \int_{-1}^{1} \rho(\xi) \ \hat{\psi}_j (\xi) \hat{\psi}_i (\xi) \frac{\abs{k_m}}{2} \ d\xi
\end{equation*}

Applying the quadrature rule:
\begin{equation}
    [M]_{ij} \approx \sum_{m=1}^{M} \sum_{k=1}^{n+1} \rho(\xi_k) \ \hat{\psi}_j (\xi_k) \hat{\psi}_i (\xi_k) \frac{\abs{k_m}}{2} \alpha_k = 
    \begin{cases}
        \sum_{m=1}^{M} \rho(\xi_i) \frac{\abs{k_m}}{2} \alpha_i, & \quad i=j \\[3pt]
        0, & \quad i \neq j
    \end{cases}
\end{equation}

\clearpage
