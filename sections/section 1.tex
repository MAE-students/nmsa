\section{Lesson 1}\label{sec:les_01}

The first lesson is an introduction to mathematical modelling, focusing on the possibilities and problems of numerical computation.

\subsection{Introduction to mathematical modelling}

The goal of this course is to obtain \textbf{numerical solutions} of continuous problems such as PDEs (partial differential equations).
A \textbf{physical problem} occurs (i.e. sound propagating in a room) and it is described through \textbf{mathematical modelling}.
The following step is to implement a \textbf{numerical algorithm} by means of a software.

\begin{figure}[H]
    \centering
    \includegraphics[width=0.6\linewidth]{math_cycle.png}
    \caption{Mathematical modelling cycle}
    
\end{figure}

Starting from a wave equation (as any other problem), we define a numerical model as $ F (\underline{x}, \underline{d}) = 0 $ (equations $ F $, solutions $ \underline{x} $, data $ \underline{d} $), which translates into the numerical algorithm $ F_n (\underline{x}_n, \underline{d}_n) = 0 $ (approximated equations $ F_n $, solutions $ \underline{x}_n $, data $ \underline{d}_n $).
The solution $ \underline{x} $ is the \textbf{analytic solution}, which differs from the approximated one $ \underline{x}_n $ called \textbf{numerical solution}.
In the end, we compare the algorithm outputs with the physical observations.

\begin{figure}[H]
    \centering
    \includegraphics[width=0.6\linewidth]{num_errors.png}
    \caption{Numerical modelling approximations}
    
\end{figure}

We start talking about the \textbf{finite element method}, introducing later on the \textbf{finite difference method} and the \textbf{finite volume method}.

The rest of the lesson has been omitted for convenience.

\clearpage
