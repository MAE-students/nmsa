\section{Lesson 12}\label{sec:les_12}

In this lesson, we want to present an alternative to finite element and finite difference methods, called finite volume method.

\subsection{From finite difference ...}

Let's consider the Burger's equation as in \eqref{eq:Burger}, with the following initial condition $u_0(x)$:
\begin{equation}
    u_0(x) =
    \begin{cases}
        -1, & x < 0 \\[3pt]
        1, & x > 0
    \end{cases}
\end{equation}
\begin{figure}[H]
    \centering
    \includegraphics[width=0.35\linewidth]{Burger_volume.png}
    \caption{Burger's equation and discontinuity $u_L < u_R$}
\end{figure}

The discontinuity $u_L < u_R$ describes a rarefaction wave scenario (defined in \eqref{eq:rarefaction}), which has an entropic solution:
\begin{equation}
    \begin{cases}
        -1, & x < - t \\[3pt]
        \dfrac{x}{t}, & - t \leq x \leq t \\[3pt]
        +1, & x > t
    \end{cases}
\end{equation}

Its temporal evolution spreads the discontinuity we have in $x=0$:
\begin{figure}[H]
    \centering
    \includegraphics[width=0.75\linewidth]{Solution_evolution.png}
    \caption{Temporal evolution at time instants $0<t'<t"$}
\end{figure}

We remind that we have infinite solutions, so we have also bad solutions, like $ u(x,t) = u_0(x) $.
The \textbf{speed of the propagation shock} is:
\begin{equation}
    \sigma = \frac{[F]}{[u]} = \frac{u_L + u_R}{2} = \frac{-1+1}{2} = 0
\end{equation}

Let's discretize the space domain $ \mathbb{R} $ using a Cartesian grid excluding the discontinuity point $x=0$.
The initial conditions state:
\begin{equation}
    u_j^0 =
    \begin{cases}
        -1, & j<0 \\[3pt]
        1, & j>0
    \end{cases}
\end{equation}
\begin{figure}[H]
    \centering
    \includegraphics[width=0.3\linewidth]{Cartesian_grid.png}
    \caption{Cartesian grid without discontinuity point and initial conditions $u_j^0$}
\end{figure}

We choose to apply an upwind scheme, in conservative form, to this problem as:
\begin{equation}
    u_j^{n+1} = u_j^n - \frac{\Delta t}{h} \left[ F(u_{j+1}, u_j) - F(u_j, u_{j-1}) \right]
\end{equation}
\begin{equation*}
    \quad F(v,w) =
    \begin{cases}
        F(v), & \sigma \geq 0 \\[3pt]
        F(w), & \sigma < 0
    \end{cases}
    , \quad \sigma = \frac{F(v) - F(w)}{v-w}
\end{equation*}

When we evaluate $F(u_{j+1}, u_j)$ we have three scenarios: both points $x_{j+1}$ and $x_j$ are positive, negative or one is positive and one is negative.
\begin{equation}
    F(v) = \frac{v^2}{2}, \quad F(u_{j+1}, u_j) =
    \begin{cases}
        F(-1,-1) = F(-1) = \frac{1}{2} \\[3pt]
        F(1,-1) = F(1) = \frac{1}{2} \\[3pt]
        F(1,1) = F(1) = \frac{1}{2}
    \end{cases}
\end{equation}

The solution computation becomes easy:
\begin{equation}
    \begin{array}{c}
        n=0 \Rightarrow u_j^1 = u_j^0 - \frac{\Delta t}{h} \cancel{\left( \frac{1}{2} - \frac{1}{2} \right)} = u_j^0 \\[3pt]
        n=1 \Rightarrow u_j^2 = u_j^1 - \frac{\Delta t}{h} \cancel{\left( \frac{1}{2} - \frac{1}{2} \right)} = u_j^1 \\[3pt]
        \Rightarrow u_j^{n+1} = u_j^n = u_j^{n-1} = \cdots = u_j^2 = u_j^1 = u_j^0, \quad \forall n, \ \forall j
    \end{array}
\end{equation}

This scheme converges to the bad solution $u_0(x)$: we want to avoid that.
We can try fixing that by including the discontinuity in the Cartesian grid:
\begin{equation}
    u_j^0 =
    \begin{cases}
        -1, & j<0 \\[3pt]
        0, & j=0 \\[3pt]
        1, & j>0
    \end{cases}
\end{equation}
\begin{figure}[H]
    \centering
    \includegraphics[width=0.35\linewidth]{Cartesian_grid_2.png}
    \caption{Cartesian grid with discontinuity point and initial conditions $u_j^0$}
\end{figure}

We want a better remedy: we are looking for schemes that satisfy the entropic condition for the non-linear hyperbolic equations \eqref{eq:non_lin_hyper}.

\subsection{... To finite volume schemes}

We need to introduce a new method because finite element and finite difference schemes are not good in proximity of discontinuities.
\textbf{Finite volume schemes} rely on the integral form of the problem and conserve the quantities (solution) at discrete level.

Let's take the problem \eqref{eq:non_lin_hyper} and integrate over the space domain $ x \in (a,b) $:
\begin{equation}
    \partial_t u + \partial_x F(u) = 0, \quad \int_a^b \partial_t u \ dx = \partial_t \left( \int_a^b u \ dx \right) = - \int_a^b \partial_x F(u) \ dx = - \left[ F(u(b)) - F(u(a)) \right]
\end{equation}

The idea behind finite volume schemes is to shift the computation from grid points (as in finite element and finite difference schemes) to cells.
\begin{figure}[H]
    \centering
    \includegraphics[width=0.6\linewidth]{Finite_volume_scheme.png}
    \caption{Finite volume cells}
\end{figure}

It is like approximating the function in a step-like fashion:
\begin{equation}
    u_i = \frac{1}{\Delta x_i} \int_{x_{i-\frac{1}{2}}}^{x_{i+\frac{1}{2}}} u \ dx \Rightarrow \int_{x_{i-\frac{1}{2}}}^{x_{i+\frac{1}{2}}} u \ dx = \Delta x_i \ u_i
\end{equation}
\begin{equation*}
    \partial_t \left( \int_{x_{i-\frac{1}{2}}}^{x_{i+\frac{1}{2}}} u \ dx \right) = \partial_t \left( \Delta x_i \ u_i \right) = F \left( u(x_{i-\frac{1}{2}}, t) \right) - F \left( u(x_{i+\frac{1}{2}}, t) \right)
\end{equation*}
\begin{equation*}
    \int_{t_n}^{t_{n+1}} \partial_t \left( \Delta x_i \ u_i \right) dt = \int_{t_n}^{t_{n+1}} F \left( u(x_{i-\frac{1}{2}}, t) \right) - F \left( u(x_{i+\frac{1}{2}}, t) \right) \ dt
\end{equation*}
\begin{equation*}
    \Delta x_i \ \left( u_i(t_{n+1}) - u_i(t_n) \right) = \int_{t_n}^{t_{n+1}} F \left( u(x_{i-\frac{1}{2}}, t) \right) - F \left( u(x_{i+\frac{1}{2}}, t) \right) \ dt
\end{equation*}
\begin{equation*}
    u_i(t_{n+1}) - u_i(t_n) = \frac{1}{\Delta x_i} \int_{t_n}^{t_{n+1}} F \left( u(x_{i-\frac{1}{2}}, t) \right) - F \left( u(x_{i+\frac{1}{2}}, t) \right) \ dt
\end{equation*}

The evolution of the cell average $u_i$ depends on the definition of the numerical flux $F(u(x,t))$.
The problem is that in $x_{i \pm \frac{1}{2}}$, as any other grid node, we have two values of $u$.
That is because we know the value of $u_i$ at the center of the cell, not at the edges.
We talk about \textbf{flux reconstruction}:
\begin{equation}
    F \left( u(x_{i-\frac{1}{2}}, t) \right) \approx F_{i-\frac{1}{2}}, \quad \partial_t \left( \sum_{i=1}^{N} u_i \ \Delta x_i \right) = \sum_{i=1}^{N} F_{i-\frac{1}{2}} - F_{i+\frac{1}{2}} = F_{\frac{1}{2}} - F_{N+\frac{1}{2}}
\end{equation}
\begin{figure}[H]
    \centering
    \includegraphics[width=0.35\linewidth]{Finite_volume_cells.png}
    \caption{Space domain discretization and finite volume cells}
\end{figure}

The scheme is \textbf{conservative} because the points computation depends only on initial and final flux.

We can define the flux as we want, resulting in different schemes, but they are all conservative.
We define the \textbf{central flux scheme}, making a Taylor expansion to compute $ u(x_{i+\frac{1}{2}}) $ using only averages values:
\begin{equation}
    u(x_{i+\frac{1}{2}} + \Delta x_{i+1}) = u_{i+1} = u(x_{i+\frac{1}{2}}) + \Delta x_{i+1} \ \partial_x u ( x_{i+\frac{1}{2}} ) + \frac{\Delta x_{i+1}^2}{2} \ \partial_{xx} u ( x_{i+\frac{1}{2}}) =
\end{equation}
\begin{equation*}
    = u_{i+\frac{1}{2}} + \Delta x_{i+1} \ \partial_x u ( x_{i+\frac{1}{2}} ) + \frac{\Delta x_{i+1}^2}{2} \ \partial_{xx} u ( x_{i+\frac{1}{2}})
\end{equation*}
\begin{figure}[H]
    \centering
    \includegraphics[width=0.25\linewidth]{Central_flux.png}
    \caption{Central flux scheme}
\end{figure}
\begin{equation*}
    u_{i+1} = \frac{1}{\Delta x_{i+1}} \int_0^{\Delta x_{i+1}} u(x_{i+\frac{1}{2}} + \Delta x_{i+1}) \ dx
\end{equation*}
\begin{equation*}
     \Rightarrow u_{i+1} = u_{i+\frac{1}{2}} + \frac{\Delta x_{i+1}}{2} \ \partial_x u + \frac{\Delta x_{i+1}^2}{6} \ \partial_{xx} u
\end{equation*}

Repeating for $u_i$:
\begin{equation}
    u_i = \frac{1}{\Delta x_i} \int_{-\Delta x_i}^0 u(x_{i+\frac{1}{2}} - \Delta x_{i+1}) \ dx = \cdots = u_{i+\frac{1}{2}} - \frac{\Delta x_i}{2} \ \partial_x u + \frac{\Delta x_i^2}{6} \ \partial_{xx} u
\end{equation}

We want to satisfy the following:
\begin{equation}
    \begin{cases}
        u_{i+\frac{1}{2}}^{-} = b \left( u_{i+1} - \dfrac{\Delta x_{i+1}}{2} \ \partial_x u - \dfrac{\Delta x_{i+1}^2}{6} \ \partial_{xx} u \right) \\[3pt]
        u_{i+\frac{1}{2}}^{+} = a \left( u_i + \dfrac{\Delta x_i}{2} \ \partial_x u - \dfrac{\Delta x_i^2}{6} \ \partial_{xx} u \right)
    \end{cases}
\end{equation}

\begin{equation*}
    \Rightarrow 2 u_{i+\frac{1}{2}} = a \ u_i + b \ u_{i+1} + \left( a \ \Delta x_i - b \ \Delta x_{i+1} \right) + \frac{1}{2} \partial_x u + \cdots \Rightarrow
    \begin{cases}
        a+b=1 \\[3pt]
        a \ \Delta x_i - b \ \Delta x_{i+1} = 0
    \end{cases}
\end{equation*}

If the grid is uniform ($ \Delta x_i = \Delta x_{i+1} $), we talk about \textbf{volume average}:
\begin{equation}
    a = b = \frac{1}{2}
\end{equation}

Otherwise, we talk about \textbf{weighted volume average}:
\begin{equation}
    a = \frac{\Delta x_{i+1}}{\Delta x_i + \Delta x_{i+1}}, \quad b = \frac{\Delta x_i}{\Delta x_i + \Delta x_{i+1}}
\end{equation}

As a result:
\begin{equation}
    u_{i+\frac{1}{2}} = a \ u_i + b \ u_{i+\frac{1}{2}}, \quad F_{i+\frac{1}{2}} =
    \begin{cases}
        F(a \ u_i + b \ u_{i+\frac{1}{2}}), & \Delta x_i \neq \Delta x_{i+1} \\[3pt]
        \frac{1}{2} F(u_i) + \frac{1}{2} F(u_{i+1}), & \Delta x_i = \Delta x_{i+1}
    \end{cases}
\end{equation}

We want now, for each cell, the profile of $u$ to be:
\begin{equation}
    \partial_t u_i = \frac{1}{\Delta x_i} \left( F_{i-\frac{1}{2}} - F_{i+\frac{1}{2}} \right)
\end{equation}

The computation consists in a time integration: we can choose the integration method we prefer.
Let's make an example with initial condition $u_0(x) = \sin (2 \pi x)$:
\begin{figure}[H]
    \centering
    \includegraphics[width=0.6\linewidth]{Finite_volume_sin.png}
    \caption{Example of initial condition discretization}
\end{figure}
\begin{equation}
    u_0^i = \frac{1}{\Delta x_i} \int_{x_{i-\frac{1}{2}}}^{x_{i+\frac{1}{2}}} \sin (2 \pi x) \ dx = - \frac{1}{2 \pi} \frac{\cos (2 \pi x_{i+\frac{1}{2}}) - \cos (2 \pi x_{i-\frac{1}{2}})}{\Delta x_i}
\end{equation}

This scheme is 2nd order accurate, by construction, but we have spurious oscillations and does not satisfy physics.
The idea is to use the upwind scheme:
\begin{equation}
    \begin{array}{l}
        \sigma>0: \quad F_{i+\frac{1}{2}} = F_i \\
        \sigma<0: \quad F_{i+\frac{1}{2}} = F_{i+1}
    \end{array}, \quad u_{i+\frac{1}{2}} =
    \begin{cases}
        u_i, & \sigma>0 \\[3pt]
        u_{i+1}, & \sigma<0
    \end{cases}
\end{equation}

This scheme in only 1st order accurate (only $a+b=1$ is satisfied), but it is conservative.
On the other hand, it may happen that:
\begin{equation}
    u_i \approx u_{i+1} \Leftrightarrow \sigma \approx \infty \Rightarrow \sigma = \partial_u F = \partial_u \left( \frac{u^2}{2} \right) = u
\end{equation}

We can do better, for example with the \textbf{Godunov scheme} (1959), where the solution $u_j^n$ (at time $t^n = n \Delta t$) is obtained through a piece-wise reconstruction:
\begin{equation}
    \tilde{u} (x, t^n) = u_j^n, \quad x_{j-\frac{1}{2}} \leq x \leq x_{j+\frac{1}{2}}
\end{equation}

We compute $\tilde{u} (x, t^{n+1})$ as solution of the Riemann problem in $[t^n, t^{n+1}]$, and then:
\begin{equation}
    u_j^{n+1} = \frac{1}{\Delta x_i} \int_{x_{j-\frac{1}{2}}}^{x_{j+\frac{1}{2}}} \tilde{u} (x, t^{n+1}) \ dx \Rightarrow u_j^{n+1} = u_j^n - \frac{\Delta t}{\Delta x_j} \left( F(u_{j+1}^n, u_j^n) - F(u_j^n, u_{j-1}^n) \right)
\end{equation}
\begin{figure}[H]
    \centering
    \includegraphics[width=1.0\linewidth]{Godunov.png}
    \caption{Visualization of Godunov scheme}
\end{figure}

The \textbf{Godunov flux} is defined through $u^*$ so that the entropy condition is satisfied:
\begin{equation}
    F(u_L, u_R) = F(u^*(u_L,u_R)) =
    \begin{cases}
        \underset{u_L \leq u \leq u_R}{\min} F(u) & u_L < u_R \\[3pt]
        \underset{u_L \leq u \leq u_R}{\max} F(u) & u_L > u_R
    \end{cases}
\end{equation}

This is a 1st order accurate scheme because the flux reconstruction is piece-wise.
We may define a 2nd order accurate scheme using a linear reconstruction (slope $s_i$):
\begin{equation}
    s_i = \frac{u_{i+1} - u_{i-1}}{2 \Delta x}, \quad u_{i+\frac{1}{2}}^{\pm} = u_i \pm \frac{\Delta x}{2} s_i, \quad F_{i+\frac{1}{2}} = F \left( u_{i+\frac{1}{2}}^{-}, u_{i+\frac{1}{2}}^{+} \right)
\end{equation}
\begin{figure}[H]
    \centering
    \includegraphics[width=0.45\linewidth]{Godunov_linear.png}
    \caption{Linear reconstruction for Godunov scheme}
\end{figure}
